\documentclass[10pt,a4paper]{article}
\usepackage[utf8]{inputenc}
\usepackage{amsmath}
\usepackage{amsfonts}
\usepackage{amssymb}
\usepackage{bm}
\newcommand{\ivec}{\vec{\imath}}
\newcommand{\jvec}{\vec{\jmath}}
\newcommand{\kvec}{\vec{k}}
\newcommand{\uvec}{\vec{u}}
\newcommand{\matrot}{\bm{R}_{\vec{u}, \theta}}
\newcommand{\mvec}{\vec{M}}
\newcommand{\mpar}{\mvec_{\|}}
\newcommand{\mper}{\mvec_{\bot}}
\newcommand{\mout}{\mvec_{\otimes}}
\newcommand{\mrot}{\mvec_{\mbox{rot}}}
\begin{document}
\title{Matrix of a rotation in the 3D space}
\author{Jean-Marc Nuzillard}
\maketitle
A rotation in a three-dimensional vector space is defined by a unitary vector $\vec{u}$
that indicates the direction of the rotation axis and by a rotation angle $\theta$.
A base of this space $B = (\ivec, \jvec, \kvec)$ constitutes a set of unitary and pairwise
orthogonal vectors.
This document shows how to write the matrix $\matrot$ in the base $B$.

The projection $\mpar$ of any vector $\mvec$ on $\uvec$ is invariant by rotation around $\uvec$.
\[ \mpar = (\mvec \cdot \uvec).\uvec\]
The vector $\mper$ defined by:
\[ \mper = \mvec - \mpar = \mvec - (\mvec \cdot \uvec).\uvec\ \]
is orthogonal to $\uvec$ and therefore to $\mpar$:
\[ (\mvec - (\mvec \cdot \uvec).\uvec) \cdot \uvec = \mvec \cdot \uvec - (\mvec \cdot \uvec).\uvec \cdot \uvec = 0 \]
because $\uvec \cdot \uvec = 1$.
$\mper$ rotates in the plane that is orthogonal to the rotation axis.
The vector $\mout$ defined by:
\[ \mout = \uvec \wedge \mper \]
it orthogonal to $\mpar$ and to $\mper$. It can be also written as 
\[ \mout = \uvec \wedge \mvec \]
because 
\[ (\mvec \cdot \uvec).\uvec \wedge \uvec = 0. \]

From the definitions of $\mpar$, $\mper$, and $\mout$:
\begin{eqnarray*}
\mper \cdot \uvec & = & 0 \\
\mout \cdot \uvec & = & 0 \\
\mper \cdot \mout & = & 0 \\
|| \mper || & = & || \mout ||
\end{eqnarray*}

Therefore, the vector
\[ \mvec = \mpar + \mper \]
is transformed by the rotation into
\begin{eqnarray*}
\mrot & = & \mpar + \cos\theta.\mper + \sin\theta.\mout \\
& = & \cos\theta.\mvec + (1 - \cos\theta).(\mvec \cdot \uvec).\uvec + \sin\theta.\uvec \wedge \mvec
\end{eqnarray*}
because $\mpar$ is invariant by rotation around $\uvec$ and $\mper$ is rotated by $\theta$ in the vector plane
defined by $\mper$ and $\mout$, two orthogonal vectors of the same norm.
A rotation by $\theta = \pi/2$ transforms $\mper$ into $\mout$ as expected.

The substitution of $\mvec$ by $\vec{\imath}$, $\vec{\jmath}$, and $\vec{k}$ provides
the content of the columns of the three terms in $\matrot$ as a function of $\theta$
and of $u_x, u_y, u_z$ (with $u_x^2 + u_y^2 + u_z^2 = 1$), the coordinates of $\uvec$.

For example, $\ivec \cdot \uvec = u_x$
so that
\[ (\ivec \cdot \uvec).\uvec = 
\begin{pmatrix}
u_x^2 \\ u_x u_y \\ u_x u_z
\end{pmatrix}
\].

As well,
\[
\uvec \wedge \ivec =
\begin{pmatrix}
u_x \\ u_y \\ u_z
\end{pmatrix}
\wedge
\begin{pmatrix}
1 \\ 0 \\ 0
\end{pmatrix}
=
\begin{pmatrix}
0 \\ u_z \\ -u_y
\end{pmatrix}
\]

Finally,
\begin{eqnarray*}
\matrot & = & \cos\theta
\begin{pmatrix}
1 & 0 & 0 \\
0 & 1 & 0 \\
0 & 0 & 1
\end{pmatrix}
+ (1 - \cos\theta)
\begin{pmatrix}
u_x^2 & u_x u_y & u_x u_z \\
u_x u_y & u_y^2 & u_y u_z \\
u_x u_z & u_y u_z & u_z^2
\end{pmatrix} \\
& & + \sin\theta
\begin{pmatrix}
0 & -u_z & u_y \\
u_z & 0 & -u_x \\
-u_y & u_x & 0
\end{pmatrix}
\end{eqnarray*}.

\hfill$\blacksquare$
\end{document}
